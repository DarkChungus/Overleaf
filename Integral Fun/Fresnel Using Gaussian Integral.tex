\documentclass{article}
\usepackage{graphicx} % Required for inserting images
\usepackage{amsmath}

\title{Fresnel Using Gaussian Integral}
\author{Prakrit Gajurel}
\date{May 2025}

\begin{document}

\maketitle

\section{Introduction}

The Fresnel integrals are defined as follows: \\
\[
\int_0^{\infty} cos(x^2) dx = \int_0^{\infty} sin(x^2) dx = \sqrt{\frac{\pi}{8}}
\]

\begin{align}
    \text{This is a pretty difficult integral to evaluate for most, but I have just found a way to evaluate it using the Gaussian integral, and I think it’s pretty cool.}
\end{align}

\section{Evaluating the Integrals}

Note that:

\[
\int_{-\infty}^{\infty} e^{-\alpha x^2}dx = \sqrt{\frac{\pi}{\alpha}}
\]

Also note, by Euler's Identity, we get the following result:

\[
e^{-ix^2} = cos(x^2) - isin(x^2)
\]

\[
\text{Therefore we get } \int_0^{\infty} e^{-ix^2} = \sqrt{\frac{\pi}{4i}}
\]

Continuing this process we get,

\[
\int_0^{\infty} cos(x^2) dx - i \int_0^{\infty} sin(x^2) dx = \sqrt{\frac{\pi}{4i}}
\]

Note the following complex result:

\[
\sqrt{i} = \frac{1}{\sqrt{2}} - i\frac{1}{\sqrt{2}}
\]

\[
\text{Then the R.H.S of our previous equation becomes } \frac{\sqrt{\pi}}{\sqrt{8}} - i\frac{\sqrt{\pi}}{\sqrt{8}}.
\]

\[
\int_0^{\infty} cos(x^2) dx - i \int_0^{\infty} sin(x^2) dx = \frac{\sqrt{\pi}}{\sqrt{8}} - i\frac{\sqrt{\pi}}{\sqrt{8}}
\]

Which would give us our final result:

\[
\boxed{\int_0^{\infty} cos(x^2) dx = \int_0^{\infty} sin(x^2) dx = \sqrt{\frac{\pi}{8}}}
\]

As needed. $\square$

\end{document}
